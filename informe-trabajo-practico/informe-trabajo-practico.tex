\documentclass[a4paper]{article}
\usepackage[utf8]{inputenc} % para poder usar tildes en archivos UTF-8
\usepackage[spanish]{babel} % para que comandos como \today den el resultado en castellano
\usepackage{a4wide} % márgenes un poco más anchos que lo usual
\usepackage[showRevisiones]{caratula}
\usepackage{xcolor}
\usepackage{listings}
\lstset{basicstyle=\ttfamily,
  showstringspaces=false,
  commentstyle=\color{red},
  keywordstyle=\color{blue}
}

\begin{document}

\materia{Organización de Computadoras 66.20}
\tipoapunte{Trabajo Práctico #0}

\fecha{\today}

\autor{Flórez Del Carpio, Christian}{91011}{chris.florez.d.c@gmail.com}
\autor{Montenegro, Josefina}{94289}{mariajosefina.mont@gmail.com}
\autor{Quino López, Julián}{94224}{julianquino2@gmail.com}

\revision{05/09/2017}{               }{Primera versión del TP}
\maketitle

\begin{abstract}
El siguiente trabajo práctico tiene como finalidad determinar, para un determinado conjunto de palabras, cuáles de ellas son palíndromos, entendiendo como palabras a aquellas compuestas por letras [A-Z], números [0-9], guiones bajos y medios, es decir, cualquier combinación posible de los anteriormente mencionados. 
\end{abstract}


\section{Introducción}
Pueden haber tres escenarios posibles, el caso en el cual el usuario ingresa archivo de entrada y salida, el caso en el que se ingresa un archivo de entrada solamente y por último el caso donde se recibe el archivo de salida. En caso de no proporcionar un archivo de texto como entrada, se requerirá ingresar el stream por entrada standard, con un máximo de 300 caracteres. Si no se especifica un archivo de salida, se mostrarán los resultados por salida standard. 


\section{Desarrollo}

El algoritmo propuesto por el grupo consiste en parsear las palabras ingresadas para luego procesar una por una y decidir si son palíndromos o no, esto se realiza ya sea desde el archivo o utilizando el stream leído por entrada standard. Si se debe leer de la entrada standard, se crea un archivo auxiliar donde se escribe lo previamente ingresado por el usuario, a fines de reutilizar el código desarrollado para el caso donde se ingresa un archivo de texto.

\subsection{Comandos para compilar y ejecutar el programa}

Se puede compilar el programa con el siguiente comando:

\begin{lstlisting}[language=bash]
  $ gcc isPalindrome.c -o tp0
\end{lstlisting}


Y luego ejecutarlo con el comando:

\begin{lstlisting}[language=bash]
  $ ./tp0 -i input.txt -o output.txt
\end{lstlisting}

En caso de sólo querer especificar el archivo de entrada, debe ejecutarse, por ejemplo, de la siguiente manera:

\begin{lstlisting}[language=bash]
  $ ./tp0 -i input.txt
\end{lstlisting}

Análogamente si se quiere ingresar un archivo de salida:

\begin{lstlisting}[language=bash]
  $ ./tp0 -o output.txt
\end{lstlisting}

\subsection{Otros comandos}

Pueden utilizarse comandos tales como help y version, de la siguiente forma:

\begin{lstlisting}[language=bash]
  $ ./tp0 -h
\end{lstlisting}

\begin{lstlisting}[language=bash]
  $ ./tp0 -V
\end{lstlisting}

\subsection{Código fuente}
\begin{lstlisting}[language=C]
#include <stdio.h>
#include <string.h>
#include <ctype.h>
#include <getopt.h>
#include <stdbool.h>
#include <stdlib.h>

#define MAXLINEA 260
#define MAXCHARS 300

// Verifica que el archivo no esté vacío
bool empty(FILE *file) {
    long savedOffset = ftell(file);
    fseek(file, 0, SEEK_END);

    if (ftell(file) == 0) {
        return true;
    }

    fseek(file, savedOffset, SEEK_SET);
    return false;
}

// Primero se valida que el archivo exista, después que no esté vacío en caso de ser archivo input
bool validFile(FILE *file, char modo, char *argopt) {
    if (file == NULL) {
        printf("El archivo %s no existe, por favor ingrese un archivo existente \n", argopt);
        return false;
    }

    if (empty(file) && modo != 'w') {
        printf("El archivo %s está vacío, por favor ingrese un archivo no vacío \n", argopt);
        return false;
    }

    printf("Se recibió el archivo %s \n", argopt);
    return true;
}

bool isPalindrome(char *palabra) {
    int posInicial, posFinal;
    posFinal = strlen(palabra) - 1;
    for (posInicial = 0; posInicial < strlen(palabra) / 2; posInicial++, posFinal--) {
        if ((toupper(*(palabra + posInicial))) != (toupper(*(palabra + posFinal)))) {
            return false;
        }
    }
    return true;
}

void seekPalindromes(char palabras[][MAXLINEA], FILE *archivo) {
    int contadorPalabra = 0;
    while (palabras[contadorPalabra][0] != '$') {
        if (isPalindrome(palabras[contadorPalabra])) {
            fputs(palabras[contadorPalabra], archivo);
            fputs("\n", archivo);
        }
        contadorPalabra++;
    }
}

void printPalindromes(FILE *archivo) {
    char bufferLinea[MAXLINEA];
    memset(&bufferLinea, 0, MAXLINEA);
    rewind(archivo);
    fgets(bufferLinea, MAXLINEA, archivo);
    printf("Las palabras palíndromas detectadas son: \n");
    while (!feof(archivo)) {
        printf("%s", bufferLinea);
        memset(&bufferLinea, 0, MAXLINEA);
        fgets(bufferLinea, MAXLINEA, archivo);
    }
}

bool validCharacter(char character) {
    int asciiNumber = (int) character;
    if ((asciiNumber <= 57) && (asciiNumber >= 48)) {
        return true;
    }
    if ((asciiNumber <= 90) && (asciiNumber >= 65)) {
        return true;
    }
    if ((asciiNumber <= 122) && (asciiNumber >= 97)) {
        return true;
    }
    if (asciiNumber == 45) {
        return true;
    }
    if (asciiNumber == 95) {
        return true;
    }
    return false;
}

void parseLine(char *linea, char palabras[][MAXLINEA]) {
    bool salir = false;
    int contador = 0;
    int contDePalabrasGuardadas = 0;
    int contDeCaracteresGuardados = 0;
    while (salir == false) {
        if (validCharacter(linea[contador])) {
            palabras[contDePalabrasGuardadas][contDeCaracteresGuardados] = linea[contador];
            contDeCaracteresGuardados++;
        } else if (contDeCaracteresGuardados != 0) {
            palabras[contDePalabrasGuardadas][contDeCaracteresGuardados] = '\0';
            contDeCaracteresGuardados = 0;
            contDePalabrasGuardadas++;
        }
        if ((linea[contador] == '\n') || (linea[contador] == '\0')) {
            salir = true;
        }
        contador++;
    }
    palabras[contDePalabrasGuardadas][0] = '$';
}

void processInput(FILE *inputFile, FILE *outputFile, bool showResultsInStdOut) {
    char bufferLinea[MAXLINEA];
    char palabras[MAXLINEA][MAXLINEA];
    // para reposicionar el puntero del archivo a la primera linea
    // lectura anticipada del archivo para q no de mas lecturas
    rewind(inputFile);
    do {
        fgets(bufferLinea, MAXLINEA, inputFile);
        parseLine(bufferLinea, palabras);  // carga en la matriz las palabras
        seekPalindromes(palabras, outputFile);
    } while (!feof(inputFile));

    fclose(inputFile);

    printf("Se procesó el archivo de entrada \n");

    if (showResultsInStdOut) {
        printPalindromes(outputFile);//usamos rewind(outputFile) para llevar el indicador de posicion del archivo a la 1era linea.
    }
    fclose(outputFile);
}


int main(int argc, char *argv[]) {
    int option = 0;
    const char *short_opt = "i:o:hV";
    struct option long_opt[] = {
            {"version", no_argument,       NULL, 'V'},
            {"help",    no_argument,       NULL, 'h'},
            {"input",   required_argument, NULL, 'i'},
            {"output",  required_argument, NULL, 'o'},
            {NULL, 0,                      NULL, 0}
    };
    FILE *inputFile = NULL;
    FILE *outputFile = NULL;
    bool takeStreamFromStdIn = false;
    bool showResultsInStdOut = false;
    char inputByStd[MAXCHARS];
    char *inputFileAux = "inputFileAux.txt";
    char *outputFileAux = "outputFileAux.txt";

    if (argc == 1) {
        printf("Debe ingresar algún argumento, para mas información ingrese -h \n");
        return 0;
    }

    while ((option = getopt_long(argc, argv, short_opt, long_opt, NULL)) != -1) {
        switch (option) {
            case 'V':
                printf("TP #0 de la materia Organización de Computadoras \n");
                printf("Alumnos: \n");
                printf("    Flórez Del Carpio Christian\n   Montenegro Josefina \n  Quino Lopez Julian \n");
                return 0;
            case 'h':
                printf("Usage: \n");
                printf("    %s -h \n", argv[0]);
                printf("    %s -V \n", argv[0]);
                printf("    %s [options] \n", argv[0]);
                printf("Options: \n");
                printf("    -V, --version  Print version and quit. \n");
                printf("    -h, --help     Print this information. \n");
                printf("    -o, --output   Location of the output file. \n");
                printf("    -i, --input    Location of the input file. \n");
                return 0;
            case 'i':
                inputFile = fopen(optarg, "r");
                if (!validFile(inputFile, 'r', optarg)) {
                    return 0;
                }
                break;
            case 'o':
                outputFile = fopen(optarg, "w");
                if (!validFile(outputFile, 'w', optarg)) {
                    return 0;
                }
                break;
            default:
                printf("Opción inválida. Para ver más información ingrese -h. \n");
        }
    }

    if (inputFile == NULL) {
        printf("Ingrese el stream a procesar (máximo 300 caracteres): \n");
        scanf("%*c%[^\n]",inputByStd);
        inputFile = fopen(inputFileAux, "w+");
        fputs(inputByStd, inputFile);
        takeStreamFromStdIn = true;
    }

    if (outputFile == NULL) {
        printf("Se mostrará el resultado en pantalla. \n");
        outputFile = fopen(outputFileAux, "w+");
        showResultsInStdOut = true;
    }

    processInput(inputFile, outputFile, showResultsInStdOut);

    // Borramos los archivos auxiliares utilizados
    if (takeStreamFromStdIn) remove(inputFileAux);
    if (showResultsInStdOut) remove(outputFileAux);

    return 0;
}
\end{lstlisting}

\section{Este es el Título de Otra Sección}

Texto de la otra sección. En la figura~\ref{fig001} se muestra un ejemplo de cómo presentar las ilustraciones del informe.

\begin{figure}[!htp]
\begin{center}
\includegraphics[width=0.5\textwidth]{fig001.eps}
\caption{Facultad de Ingeniería $-$ Universidad de Buenos Aires.} \label{fig001}
\end{center}
\end{figure}


\subsection{Este es el Título de Otra Subsección}

Texto de la otra subsección...


\section{Conclusiones}

Se presentó un modelo para que los alumnos puedan tomar como referencia en la redacción de sus informes de trabajos prácticos.


\begin{thebibliography}{99}

\bibitem{INT06} GetOpt library, https://www.gnu.org/software/libc/manual/html_node/Example-of-Getopt.html

\bibitem{HEN00} J. L. Hennessy and D. A. Patterson, ``Computer Architecture. A Quantitative
Approach,'' 3ra Edición, Morgan Kaufmann Publishers, 2000.

\bibitem{LAR92} J. Larus and T. Ball, ``Rewriting Executable Files to Mesure Program Behavior,'' Tech. Report 1083, Univ. of Wisconsin, 1992.

\end{thebibliography}

\end{document}
